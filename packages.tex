%           README. 
% **Edits respect this:**
% =======================
%
% Packages in two *alphabetical* sections to avoid double packages.
% First section with packages that need further configuration,
% then packages which are "one liners".
%
% +---------------------------------------------------------------------+
% > Before inserting package, check if it is in one of both categories! | 
% +---------------------------------------------------------------------+
% 
% ## General style:
% *Comment* for every package mandatory. (Thank me later)
% REQ: Package requires another package
% BC: Package is installed because of another package
% OPT: List packages which are optional for this package
%
% ## Config packages:
% *Newline* between packages. Configuration comes right after install
% --------------------------------------------------------------------

% ## Open Issues:
% - Comment for every package
% - remove packages with identical function
% - remove deprecated packages
% - Old comments which are not clear by now:
% Do #locale to tag locale settings
% Die Anpassung an Landessprache verwendet globale Option german siehe \documentclass
% -------------------------------------------------------------
% Befehle aus AMSTeX für mathematische Symbole z.B. \boldsymbol \mathbb ----



% -------------------------------------------------------------------------
% # First packages which need config. Packages without config later in this doc
% -------------------------------------------------------------------------

%Zur Verwendung von BibLaTex
\usepackage[
    backend=biber,
	%style=alphabetic
    style=numeric,
]{biblatex}
\addbibresource{bibliografie.bib}
\DefineBibliographyStrings{ngerman}{ 
	andothers = {{et\,al\adddot}},             
}

\usepackage{caption} %?
\captionsetup{
	format = plain,
	justification = centering,
	labelsep = newline,
	singlelinecheck = false,
	labelfont = bf,
	font = small
}

\usepackage{color} % BC listings. Necessary for colored code
% --------- Code color scheme (e.g. Bash, C++) --------- 
\definecolor{backgroundcolor}{RGB}{237, 241, 237}
\definecolor{commentcolor}{RGB}{0,128,0}
\definecolor{keywordcolor}{RGB}{136, 134, 58}
\definecolor{identifiercolor}{RGB}{24, 79, 110}
\definecolor{numbercolor}{RGB}{128,128,128}
% \definecolor{numbercolor}{RGB}{0,0,0}
\definecolor{preprocessorcolor}{RGB}{128,128,128}
\definecolor{stringcolor}{RGB}{163,21,21}


\usepackage{graphicx} % grafiken. draft/final possible. 
%\graphicspath{{Grafiken/}} 	% Dort liegen die Bilder des Dokuments

\usepackage[
bookmarks,
bookmarksopen=true,
bookmarksnumbered=true,
colorlinks=true,
%linkcolor=red, % einfache interne Verknüpfungen
%anchorcolor=black,% Ankertext
%citecolor=blue, % Verweise auf Literaturverzeichniseinträge im Text
%filecolor=magenta, % Verknüpfungen, die lokale Dateien öffnen
%menucolor=red, % Acrobat-Menüpunkte
%urlcolor=cyan, 
% für die Druckversion können die Farben ausgeschaltet werden:
linkcolor=black, % einfache interne Verknüpfungen
anchorcolor=black,% Ankertext
citecolor=black, % Verweise auf Literaturverzeichniseinträge im Text
filecolor=black, % Verknüpfungen, die lokale Dateien öffnen
menucolor=black, % Acrobat-Menüpunkte
urlcolor=black, 
%backref,
%pagebackref,
plainpages=false,% zur korrekten Erstellung der Bookmarks
pdfpagelabels,% zur korrekten Erstellung der Bookmarks
hypertexnames=false,% zur korrekten Erstellung der Bookmarks
linktoc=all%Sowohl Seitenzahl als auch Text als Link
]{hyperref}

\usepackage{listings} % Code snippets. Opt: Color
% -------- Define Bash language style, escape char -------------
\lstdefinestyle{bash}{
        language=bash,
        backgroundcolor=\color{backgroundcolor},   
    commentstyle=\color{commentcolor},
        %directivestyle=\color{preprocessorcolor},
    keywordstyle=\color{keywordcolor},
        %identifierstyle=\color{identifiercolor},
    numberstyle=\footnotesize\color{numbercolor}\numberwithprompt,
    stringstyle=\color{stringcolor},
    basicstyle=\ttfamily\footnotesize,
    breakatwhitespace=false,         
    breaklines=true,                 
    captionpos=b,                    
    keepspaces=true,                 
    numbers=left,                    
    numbersep=5pt,                  
    showspaces=false,                
    showstringspaces=false,
    showtabs=false,                  
    tabsize=2,
    xleftmargin=2.5ex, % to move $ sign inside text margin
}
\newcommand{\lstprompt}{ \$} % shows a prompt in front of bash code, conforming https://developers.google.com/style/code-samples
\newcommand\numberwithprompt[1]{\color{numbercolor}#1 \lstprompt}
\lstset{escapechar=`,style=bash}
% ------------- Define sql Language style, escape chars etc. --------------
\lstdefinestyle{sql}{
    language=SQL,
	backgroundcolor=\color{backgroundcolor},   
    commentstyle=\color{commentcolor},
	%directivestyle=\color{preprocessorcolor},
    keywordstyle=\color{keywordcolor},
	identifierstyle=\color{identifiercolor},
    numberstyle=\tiny\color{numbercolor},
    stringstyle=\color{stringcolor},
    basicstyle=\ttfamily\footnotesize,
    breakatwhitespace=false,         
    breaklines=true,                 
    captionpos=b,                    
    keepspaces=true,                 
    numbers=left,                    
    numbersep=5pt,                  
    showspaces=false,                
    showstringspaces=false,
    showtabs=false,                  
    tabsize=2
}
\lstset{escapechar=@,style=sql}
\renewcommand{\lstlistingname}{Code} % Damit Codebeispiele mit Code betitelt werden.




\usepackage{paralist} % Formatierung von Listen ändern
\setdefaultleftmargin{2.5em}{2.2em}{1.87em}{1.7em}{1em}{1em} % Standardeinstellungen:

\usepackage{pgfplots} % Pdf optionen. Für Kompatibilät unter verschiedenen OS
\pgfplotsset{compat=1.7}
\pdfminorversion=7 % Verwende pdf Version 1.7

\usepackage[headsepline,automark]{scrlayer-scrpage} % Anpassung des Seitenlayouts, siehe Seitenstil.tex

\usepackage[
locale=DE, 
%per=symbol-or-fraction
]{siunitx} % Zur Richtigen Verwendung von Einheiten
\sisetup{
	mode = text,
	detect-family,
	detect-weight,  
	exponent-product = \cdot,
	output-decimal-marker={\text{.}},
	%math-rm=\mathsf,
	%text-rm=\sffamily,
	range-phrase = { -- },
	per-mode=symbol,
	per-symbol=/
}
\DeclareSIUnit{\var}{var}

\usepackage{tocbasic}
\DeclareTOCStyleEntry[
  beforeskip=.0em,% default is 1em plus 1pt
  pagenumberformat=\textbf
]{tocline}{chapter}


% ------------------------------------------
% # Packages without further config go here 
% -------------------------------------------

\usepackage[printonlyused, smaller]{acronym} %?
\usepackage{ae}			 	% "schöneres" ä #locale
\usepackage{amsmath} % better math symbols?
\usepackage{amsfonts}
%\usepackage{amsmath,amssymb,units} units macht bei mir Fehler
\usepackage{array} % für lange Tabellen
\usepackage[ngerman]{babel} %german #locale
\usepackage{blindtext} % Verwendung von Blindtext zur Layout gestaltung 
\usepackage{bookmark}
\usepackage{booktabs} % BC listing?
\usepackage{charter} % font?
\usepackage{colortbl} %farbige hinterlegung
\usepackage{courier} % font?
\usepackage[german=quotes]{csquotes} % ? #locale
\usepackage{diagbox} % für lange Tabellen
\usepackage{eurosym} % BC listing?
\usepackage{float}
\usepackage{floatflt} % Zum Umfließen von Bildern 
\usepackage[T1]{fontenc} %umlaute? #locale
\usepackage{geometry} % Einfache Definition der Zeilenabstände und Seitenränder 
\usepackage[acronyms,toc]{glossaries} %Glossar Achtung Perl.exe muss installiert werden
\usepackage{glossaries-extra} % BC glossaries
\usepackage{glossary-mcols} % BC glossaries
\usepackage{helvet} % font?
\usepackage[all]{hypcap} % jump to top of figure
\usepackage[utf8]{inputenc} %german locale
\usepackage{longtable} % für lange Tabellen
\usepackage{lscape} % für lange Tabellen
\usepackage{makeidx} % Für Index-Ausgabe; \printindex 
\usepackage{mathptmx} % font?
\usepackage{multicol} %?
\usepackage{multirow} % für lange Tabellen
\usepackage{pdfpages}
\usepackage{pifont}% http://ctan.org/pkg/pifont code?
\usepackage{placeins} %?
\usepackage{ragged2e} % für lange Tabellen
\usepackage{rotating} % für lange Tabellen
\usepackage{scrhack} %german locale
\usepackage{setspace} % Einfache Definition der Zeilenabstände und Seitenränder 
\usepackage{subfig}
\usepackage{tensor} % math?
\usepackage{textcomp} 		% Euro-Zeichen etc. #locale
\usepackage{url} % Lange URLs umbrechen etc.
\usepackage{wrapfig} %math?
\usepackage{xcolor} % DUP? color
\usepackage{xspace}

% Some config??? 
\newcolumntype{M}[1]{>{\centering\arraybackslash}m{#1}} % für lange Tabellen
\newcolumntype{d}[1]{>{\raggedleft\hspace{0pt}}p{#1}} % Spaltendefinition rechtsbündig mit definierter Breite auf dem Deckblatt 