\documentclass[
	12pt,						 % Schriftgröße
	DIV=10,						 % Satzspiegelgröße Seitenränder
	german,						 % für Umlaute, Silbentrennung etc.
	paper=a4,					 % Papierformat
	oneside, 
	titlepage,					 % es wird eine Titelseite verwendet
	parskip=half,				 % Abstand zwischen Absätzen (halbe Zeile)
	headings=normal,			 % Größe der Überschriften verkleinern
	listof=totoc,				 % Verzeichnisse im Inhaltsverzeichnis aufführen
	bibliography=totoc,			 % Literaturverzeichnis im Inhaltsverzeichnis aufführen
	numbers=noenddot,			 % Kapitelnummern ohne Punkt	
	enabledeprecatedfontcommands % Alte Befehle zur unterstüzung von \listofsymbols	
]{scrreprt}

\makeatletter	
	\setlength\@fptop{0\p@}  % Entfernt Lücken am Seitenbeginn
\makeatother

\newcommand{\titel}{Entwurf einer Datenbank für ein Verwaltungssystem von Lebensmitteln in einem Kühlschrank}
\newcommand{\art}{Ausarbeitung}
\newcommand{\autor}{Claas Meints}
\newcommand{\matrikel}{172489}
\newcommand{\semester}{6}
\newcommand{\studienbereich}{Elektrotechnik/Mechatronik}
\newcommand{\studiengang}{Elektrotechnik}
\newcommand{\erstgutachter}{Prof. Dr. Herwig Henseler}
\newcommand{\jahr}{2022}
\newcommand{\comment}[1]{}
\newcommand{\TODO}{}
\newcommand\fig[2]{
	\begin{figure}[ht]
		\centering
		\includegraphics[width=0.8\textwidth]{figures/#1}
		\caption{#2} 
		\label{fig:#1}
	\end{figure}
}
% Zur Richtigen Verwendung von Einheiten
\usepackage[
    locale=DE, 
    %per=symbol-or-fraction
]{siunitx}

\sisetup{
    mode = text,
    detect-family,
    detect-weight,  
    exponent-product = \cdot,
    output-decimal-marker={\text{.}},
    %math-rm=\mathsf,
    %text-rm=\sffamily,
    range-phrase = { -- },
    per-mode=symbol,
    per-symbol=/
}
\DeclareSIUnit{\var}{var}

% Anpassung des Seitenlayouts -------------------------------------------------------------% 	siehe Seitenstil.tex
% -------------------------------------------------------------
\usepackage[headsepline,automark]{scrlayer-scrpage}

\usepackage{listings}
\usepackage{xcolor}
% Overall Colors

% VS Code Light+ Theme
% Background: #FFFFFF
\definecolor{background}{HTML}{FFFFFF}
% Foreground: #000000
\definecolor{foreground}{HTML}{000000}
% Comment: #008000
\definecolor{comment}{HTML}{008000}
% String Literal: #A31515
\definecolor{string}{HTML}{A31515}
% Number Literal: #098658
\definecolor{number}{HTML}{098658}
% Keyword: #0000FF
\definecolor{keyword}{HTML}{0000FF}
% Function: #795E26
\definecolor{function}{HTML}{795E26}
% Type: #267F99
\definecolor{type}{HTML}{267F99}
% Control Keyword: #AF00DB
\definecolor{control}{HTML}{AF00DB}
% Decorator: #795E26
\definecolor{decorator}{HTML}{795E26}
% Variable: #001080
\definecolor{variable}{HTML}{001080}

% Define TypeScript Language
\lstdefinelanguage{typescript}{
    backgroundcolor=\color{background},
    keywords={abstract, as, assert, async, class, const, debugger, declare, enum, extends, false, function, get, implements, in, infer, instanceof, interface, is, keyof, let, module, namespace, never, new, null, of, package, private, protected, public, readonly, require, return, set, static, super, symbol, this, true, type, typeof, undefined, unique, unknown, var},
    keywordstyle=\color{keyword},
    comment=[l]{//},
    morecomment=[s]{/*}{*/},
    commentstyle=\color{comment},
    morestring=[b]",
    morestring=[b]',
    stringstyle=\color{string},
    identifierstyle=\color{variable},
    classoffset=1,
    morekeywords={await, break, case, catch, continue, default, delete, do, else, export, finally, for, from, if, import, switch, throw, try, while, with, yield},
    keywordstyle=\color{control},
    classoffset=2,
    morekeywords={any, bigint, boolean, never, number, object, string, symbol, unknown, void, Model, fridge_user},
    keywordstyle=\color{type},
    morecomment=[l]{=>},
    commentstyle=\color{type},
    classoffset=3,
    morekeywords={@Table, @PrimaryKey, @Column, @BeforeCreate, @BeforeUpdate, @HasMany, hash},
    keywordstyle=\color{decorator},
    classoffset=4,  
    morekeywords={ 0, 1, 2, 3, 4, 5, 6, 7, 8, 9},
    keywordstyle=\color{number},
    classoffset=0,
}

% Listing Style
\lstdefinestyle{vscode-light+}{
    basicstyle=\ttfamily\footnotesize,
    breakatwhitespace=false,
    breaklines=true,
    captionpos=b,
    commentstyle=\color{foreground},
    deletekeywords={...},
    escapeinside={\%*}{*)},
    extendedchars=true,
    frame=single,
    keepspaces=true,
    numbers=left,
    numbersep=5pt,
    numberstyle=\color{foreground},
    language=typescript,
    tabsize=2,
}

% Set Listing Style
\lstset{style=vscode-light+, language=typescript}


\usepackage{booktabs}
\usepackage{eurosym}
\usepackage{pifont}% http://ctan.org/pkg/pifont

\renewcommand{\lstlistingname}{Code} % Damit Codebeispiele mit Code betitelt werden.


% Anpassung an Landessprache -------------------------------------------------------------% 	Verwendet globale Option german siehe \documentclass
% -------------------------------------------------------------
\usepackage[ngerman]{babel}
\usepackage{scrhack}
\usepackage[utf8]{inputenc}
\usepackage{amsmath,amssymb,units}
\usepackage{wrapfig,caption, placeins}
\usepackage{multicol}
\captionsetup{
    format = plain,
    justification = centering,
    labelsep = newline,
    singlelinecheck = false,
    labelfont = bf,
    font = small
}
\usepackage{mathptmx, times, helvet, courier} % Pakete für Schriftarten
%\usepackage{mathptmx,charter,helvet,courier}
\usepackage[printonlyused, smaller]{acronym}
\usepackage{wrapfig}
\usepackage{pgfplots}
% Umlaute -------------------------------------------------------------% 		Umlaute/Sonderzeichen wie äöüß direkt im Quelltext verwenden (CodePage).
%		Erlaubt automatische Trennung von Worten mit Umlauten.
% -------------------------------------------------------------
\usepackage[utf8]{inputenc}
\usepackage[T1]{fontenc}
\usepackage{ae}			 	% "schöneres" ä
\usepackage{textcomp} 		% Euro-Zeichen etc.
\usepackage[german=quotes]{csquotes}


% Verwendung von Blindtext zur Layout gestaltung -------------------------------------------------------------
\usepackage{blindtext}

% Grafiken -------------------------------------------------------------
% 		Einbinden von Grafiken [draft oder final]
% 		Option [draft] bindet Bilder nicht ein - auch globale Option
% -------------------------------------------------------------
\usepackage{graphicx}
%\graphicspath{{Grafiken/}} 	% Dort liegen die Bilder des Dokuments


% Befehle aus AMSTeX für mathematische Symbole z.B. \boldsymbol \mathbb ----
\usepackage{amsmath,amsfonts}
\usepackage{tensor}

% Für Index-Ausgabe; \printindex -------------------------------------------------------------
\usepackage{makeidx}

% Einfache Definition der Zeilenabstände und Seitenränder -------------------------------------------------------------
\usepackage{setspace}
\usepackage{geometry}


%% Zum Umfließen von Bildern -------------------------------------------------------------
\usepackage{floatflt}


% Lange URLs umbrechen etc. -------------------------------------------------------------
\usepackage{url}

% für lange Tabellen
\usepackage{longtable}
\usepackage{multirow}
\usepackage{rotating}
\usepackage{diagbox}
\usepackage{array}
\newcolumntype{M}[1]{>{\centering\arraybackslash}m{#1}}
\usepackage{ragged2e}
\usepackage{lscape}
\usepackage{colortbl} %farbige hinterlegung


% Spaltendefinition rechtsbündig mit definierter Breite auf dem Deckblatt 
%------------------------------------------------------------
\newcolumntype{d}[1]{>{\raggedleft\hspace{0pt}}p{#1}}

% Formatierung von Listen ändern
\usepackage{paralist}
% Standardeinstellungen:
\setdefaultleftmargin{2.5em}{2.2em}{1.87em}{1.7em}{1em}{1em}

%Zur Verwendung von BibLaTex
\usepackage[
    style=alphabetic
]{biblatex}
\addbibresource{bibliografie.bib}
\DefineBibliographyStrings{ngerman}{
    andothers = {{et\,al\adddot}},             
}

\usepackage{caption}
\usepackage{float}
\usepackage{subfig}
\usepackage{xspace}
\usepackage{scrhack}

\usepackage[acronyms,toc]{glossaries} %Glossar Achtung Perl.exe muss installiert werden
\usepackage{glossaries-extra}
\usepackage{glossary-mcols}

% pdf-Optionen  -------------------------------------------------------------
\usepackage{pgfplots} % Für Kompatibilät unter verschiedenen OS
\pgfplotsset{compat=1.7}

\pdfminorversion=7 % Verwende pdf Version 1.7

\usepackage[
    bookmarks,
    bookmarksopen=true,
    bookmarksnumbered=true,
    colorlinks=true,
    %linkcolor=red, % einfache interne Verknüpfungen
    %anchorcolor=black,% Ankertext
    %citecolor=blue, % Verweise auf Literaturverzeichniseinträge im Text
    %filecolor=magenta, % Verknüpfungen, die lokale Dateien öffnen
    %menucolor=red, % Acrobat-Menüpunkte
    %urlcolor=cyan, 
    % für die Druckversion können die Farben ausgeschaltet werden:
    linkcolor=black, % einfache interne Verknüpfungen
    anchorcolor=black,% Ankertext
    citecolor=black, % Verweise auf Literaturverzeichniseinträge im Text
    filecolor=black, % Verknüpfungen, die lokale Dateien öffnen
    menucolor=black, % Acrobat-Menüpunkte
    urlcolor=black, 
    %backref,
    %pagebackref,
    plainpages=false,% zur korrekten Erstellung der Bookmarks
    pdfpagelabels,% zur korrekten Erstellung der Bookmarks
    hypertexnames=false,% zur korrekten Erstellung der Bookmarks
    linktoc=all%Sowohl Seitenzahl als auch Text als Link
]{hyperref}
\usepackage{bookmark}
\usepackage[all]{hypcap} % jump to top of figure
\usepackage{pdfpages}
\input{style/pagestyle.tex}

\newglossary[llg]{glossary}{llo}{lls}{Glossar}
\makeglossaries

\begin{document}
	\input{style/titlepage.tex}
	
	\pagenumbering{Roman}
	\setcounter{tocdepth}{2} % Tiefe des Inhaltsverzeichnisses definiert
	\tableofcontents

    \chapter{Einleitung}\label{ch:Einleitung}
\pagenumbering{arabic}
\section{Motivation}\label{sec:Motivation}
- IoT ist ein großes Thema\\
- Kennzahlen für IoT\\
\section{Ziel der Arbeit}\label{sec:Ziel der Arbeit}
- Ausführlicher Vergleich von IoT-Plattformen mit Fokus auf die Entwicklung von IoT-Lösungen vom Prototyp bis zum skalierten Geschäftsmodell\\
- Auswahl geeigneter Kriterien anhand eines Referenzmodells mit Rückblick auf Prototypentwicklung und Skalierung\\
\section{Aufbau der Arbeit}\label{sec:Aufbau der Arbeit}

    \chapter{Stand der Technik}\label{ch:Stand der Technik}
\section{Komponenten einer IoT-Lösung}\label{sec:Komponenten einer IoT-Lösung}
% [Guth2018] [10.1007/978-3-030-24311-1_8]\\
\subsection{Geräte}\label{subsec:Geräte}
- Sensoren und Aktoren\\
- Treiber für Sensoren und Aktoren\\
- Geräte SDKs\\
- RTOS\\
- Edge Computing\\
\subsection{Gateway}\label{subsec:Gateway}
- Software und Schnittstellen für Gerätekommunikation\\
- Gateway SDKs\\
- OS\\
- Schnittstellen zu Cloud-Plattformen (MQTT, REST, etc.)\\
- Fog Computing\\
\subsection{IoT-Software-Plattform}\label{subsec:IoT-Software-Plattform}
- Verwaltet Geräte und Gateways\\
- Stellt Schnittstellen für externe Applikationen und Services bereit\\
- bietet Services für die Verarbeitung von Daten\\
- wird in dieser Arbeit als IoT-Plattform bezeichnet\\
\subsection{Anwendung}\label{subsec:Anwendung}
- Anwendungen / Applikationen / Services auf Basis der IoT-Plattform und der Daten der Geräte\\
\section{Referenzarchitektur von IoT-Software-Plattformen}\label{sec:Referenzarchitektur von IoT-Software-Plattformen}
\subsection{Kommunikation}\label{subsec:Kommunikation}
- MQTT Broker\\
- (MQTT-)WebSockets\\
- HTTP (REST)\\
\subsection{Digitaler Zwilling}\label{subsec:Digitaler Zwilling}
- Device Registry / Twin\\
- Device Metadata and Data\\
- Remote Actions\\
- Device Updates\\
\subsection{IoT-Services}\label{subsec:IoT-Services}
% [Lempert2021] [Microsoft_2021] [9825678]\\
- Develop Insights (Cold, Warm and Hot Path)\\
- Databases (Structured, Unstructured and Time Series)\\
- Descriptive Analytics\\
- Diagnostic Analytics\\
- Real-Time Analytics (Stream Processing Hot Path)\\
- Predictive / Prescriptive Analytics (Machine Learning Cold Path)\\
\subsection{Integration in Drittsysteme}\label{subsec:Integration in Drittsysteme}
- Adapter zu Enterprise-Systemen\\
- Cloud to Cloud Integration\\
- Apis?\\
\subsection{Orchestrierung von Services}\label{subsec:Orchestrierung von Services}
- Message Broker / Event Bus\\
- Rules Engine\\
- Service-Orchestrierung\\
- Load Balancing?\\
\subsection{Management}\label{subsec:Management}
- Device Management -> Device Twin\\
- Configuration / State Management -> Device Twin\\
- Errors and Reports\\
\subsection{Security}\label{subsec:Security}
- Authentication / Authorization\\
- Identity Management\\
\section{Aktuelle Implementierungen von IoT-Software-Plattformen}\label{sec:Aktuelle Implementierungen von IoT-Software-Plattformen}
\subsection{Hyperscaler-Cloud-Anbieter}\label{subsec:Hyperscaler-Cloud-Anbieter}
- Bindung an den Cloud-Anbieter?\\
- Regional verteilbar -> geringere Latenz?\\
\subsection{(Open-Source IoT-Plattformen)}\label{subsec:(Open-Source IoT-Plattformen)}
- Cloud agnostisch?\\
- On-Premise möglich?\\
\subsection{(IoT-Spezifische Plattformen)}\label{subsec:(IoT-Spezifische Plattformen)}
% ^ möglicherweise zusammenfassen ^\\
\subsection{Low-Code-Plattformen}\label{subsec:Low-Code-Plattformen}
- schnellere Entwicklung von IoT-Lösungen?\\
- höhere Flexibilität?\\
- hohe Abhängigkeit von der Plattform?\\
- hohe Kosten?\\
\section{IoT Geschäftsmodellentwicklung}\label{sec:IoT Geschäftsmodellentwicklung}
\subsection{Prototypentwicklung}\label{subsec:Prototypentwicklung}
\subsection{Skalierung}\label{subsec:Skalierung}

    \chapter{Theoretische Grundlagen}\label{ch:Theoretische Grundlagen}
\section{(Übertragungstechnologien)}\label{sec:(Übertragungstechnologien)}
\subsection{(WLAN)}\label{subsec:(WLAN)}
\subsection{(Mobilfunk)}\label{subsec:(Mobilfunk)}
\subsection{(LoRaWAN)}\label{subsec:(LoRaWAN)}
\section{(Übertragungsprotokolle und Schnittstellen)}\label{sec:(Übertragungsprotokolle und Schnittstellen)}
\subsection{(MQTT)}\label{subsec:(MQTT)}
- Publish / Subscribe\\
- MQTT WebSockets\\
\subsection{(HTTP)}\label{subsec:(HTTP)}
- REST\\

    \chapter{Anforderungsdefiniton (An die Arbeit)}\label{ch:Anforderungsdefiniton (An die Arbeit)}
\section{Entwicklung von Kriterien zur Auswahl einer IoT-Software-Plattform}\label{sec:Entwicklung von Kriterien zur Auswahl einer IoT-Software-Plattform}
\section{Vergleich der technischen Funktionalitäten von IoT-Software-Plattformen}\label{sec:Vergleich der technischen Funktionalitäten von IoT-Software-Plattformen}
\section{Vergleich nicht-funktionaler Anforderungen an IoT-Software-Plattformen}\label{sec:Vergleich nicht-funktionaler Anforderungen an IoT-Software-Plattformen}
\section{Betrachtung von der Prototyp-Phase bis zur skalierten Lösung}\label{sec:Betrachtung von der Prototyp-Phase bis zur skalierten Lösung}
\section{(Evaluierung anhand eines Beispielprojekts)}\label{sec:(Evaluierung anhand eines Beispielprojekts)}
\section{Abgrenzung der Arbeit}\label{sec:Abgrenzung der Arbeit}

    \chapter{Systemanalyse (Grundlage für Kriterien)}\label{ch:Systemanalyse (Grundlage für Kriterien)}
\section{Kernaufgaben von IoT-Software-Plattformen}\label{sec:Kernaufgaben von IoT-Software-Plattformen}
- Anforderungen aus Referenzarchitektur ableiten\\
- Welche eigenschaften haben die am weitesten verbreiteten IoT-Plattformen?\\
\section{Erfolgsfaktoren für IoT-Geschäftsmodelle}\label{sec:Erfolgsfaktoren für IoT-Geschäftsmodelle}
- schnelle Prototypentwicklung (Time to Market)\\
- hohe Flexibilität\\
- integration in bestehende Systeme der Firma\\
- Nutzbarkeit von Synergien mit anderen Geschäftsmodellen\\
- Skalierbarkeit der Kosten\\
- technische Funktionalitäten (Projektbezogen)\\

    \chapter{Konzeption (Kriterien für Prototy und Skalierte Lösung)}\label{ch:Konzeption (Kriterien für Prototy und Skalierte Lösung)}
\section{Funktionale Anforderungen}\label{sec:Funktionale Anforderungen}
\subsection{Verfügbarkeit}\label{subsec:Verfügbarkeit}
- Redundancy\\
- Fault Resilience\\
- Regional Distribution\\
- Self Healing\\
\subsection{Performanz}\label{subsec:Performanz}
- Latency\\
- Data Processing / Analytics\\
\subsection{Unterstützte Funktionen und Schnittstellen}\label{subsec:Unterstützte Funktionen und Schnittstellen}
- Communication Technologies\\
- Open Adapters / APIs\\
- Implementation of optional features\\
\subsection{Sicherheit}\label{subsec:Sicherheit}
- Support for Security-Hardware\\
\subsection{Skalierbarkeit}\label{subsec:Skalierbarkeit}
\section{Nicht-funktionale Anforderungen}\label{sec:Nicht-funktionale Anforderungen}
\subsection{Kosten und Lizenzierung}\label{subsec:Kosten und Lizenzierung}
- Kosten pro Gerät\\
- Kosten per Nutzung\\
- Lizenzkosten\\
- Open Source?\\
\subsection{Rechtliche Anforderungen}\label{subsec:Rechtliche Anforderungen}
- Datenschutz\\
- Datentransfer (EU-US Privacy Shield)\\
- Kritis?\\
\subsection{Entwicklungsaufwand und -geschwindigkeit}\label{subsec:Entwicklungsaufwand und -geschwindigkeit}
- Welche Vorkenntnisse sind nötig?\\
- Wie viel Aufwand wird benötigt um einen einfachen Prototypen zu entwickeln?\\
- Wie leicht lassen sich neue Funktionen implementieren?\\
- Wie leicht lässt sich die Lösung anpassen oder migrieren?\\
- Developer Experience\\
\section{Bewertung der Anforderungen}\label{sec:Bewertung der Anforderungen}
\subsection{Kritische Anforderungen}\label{subsec:Kritische Anforderungen}
- zwingend notwendig für die Entwicklung einer IoT-Lösung\\
- gibt es zusammengesetzte kritische Anforderungen?\\
- welche Zustände sind für Projekte unakzeptabel? (in der Prototypen- und Skalierungsphase)\\
\subsection{Gewichtung der Anforderungen}\label{subsec:Gewichtung der Anforderungen}
- Wie wichtig sind die Anforderungen für das Projekt?\\
- Ändern sich Gewichtungen im Laufe des Projekts?\\

    \chapter{Implementierung (Abgleich der Kriterien)}\label{ch:Implementierung (Abgleich der Kriterien)}
\section{Vorauswahl von IoT-Software-Plattformen}\label{sec:Vorauswahl von IoT-Software-Plattformen}
- Können Plattformen oder Plattformtypen ausgeschlossen werden?\\
- Welches sind die wichtigsten und relevantesten IoT-Plattformen?\\
- Welche IoT-Plattformen sind für die EWE besonders interessant?\\
- Welche IoT-Plattformen sind für die Entwicklung  digitaler Geschäftsmodelle besonders interessant?\\
\section{Analyse der IoT-Plattformen}\label{sec:Analyse der IoT-Plattformen}
\subsection{AWS Core IoT}\label{subsec:AWS Core IoT}
\subsection{Azure IoT}\label{subsec:Azure IoT}
\subsection{Google Cloud IoT}\label{subsec:Google Cloud IoT}
\subsection{(Digimondo nIoTx)}\label{subsec:(Digimondo nIoTx)}
\subsection{(Zenner IoT)}\label{subsec:(Zenner IoT)}
\section{Vergleich der IoT-Plattformen}\label{sec:Vergleich der IoT-Plattformen}

    \chapter{Umsetzung eines exemplarischen IoT-Geschäftsmodells}\label{ch:Umsetzung eines exemplarischen IoT-Geschäftsmodells}
\section{Auswahl einer IoT-Plattform für ein exemplarisches Geschäftsmodell}\label{sec:Auswahl einer IoT-Plattform für ein exemplarisches Geschäftsmodell}
\subsection{Auswahl eines exemplarischen Geschäftsmodells (groß)}\label{subsec:Auswahl eines exemplarischen Geschäftsmodells (groß)}
\subsection{Auswahl einer IoT-Plattform anhand des Kriterienkatalogs}\label{subsec:Auswahl einer IoT-Plattform anhand des Kriterienkatalogs}
\subsection{Evaluation des Auswahlprozesses anhand des Geschäftsmodells}\label{subsec:Evaluation des Auswahlprozesses anhand des Geschäftsmodells}
\section{Prototypische Implementierung des Geschäftsmodells}\label{sec:Prototypische Implementierung des Geschäftsmodells}
\subsection{Umsetzung einer Teilfunktionalität auf der IoT-Plattform (klein)}\label{subsec:Umsetzung einer Teilfunktionalität auf der IoT-Plattform (klein)}
\subsection{Simulation des Betriebs der skalierten Lösung}\label{subsec:Simulation des Betriebs der skalierten Lösung}
\subsection{Evaluation der Umsetzung}\label{subsec:Evaluation der Umsetzung}
- optionale Funktionen / extras\\
- Sicherheit\\
- Skalierbarkeit\\
- Kosten wie erwartet?\\
- Kostenentwicklung wie erwartet?\\
- Kostenloser Einstieg?\\
- Developer Experience\\
- Vorkenntnisse wie erwartet?\\
- Bedienbarkeit\\
- Performanz (Global vs. Regional)\\
- Analysen\\
- Anzahl der Geräte\\
- Anzahl der Anwendungen\\
- Kosten für Performanz\\
- Fehler?\\
- Wie verhält sich das System bei Ausfällen?\\
- Bewusstes zerstören -> wie verhält sich das System?\\
- Wie verhält sich das System bei Lastspitzen?\\
- Regeneriert sich das System selbst?\\

    \chapter{Ergebnis und Diskussion}\label{ch:Ergebnis und Diskussion}
\section{Anwendungsspezifische Bewertung von IoT-Plattformen}\label{sec:Anwendungsspezifische Bewertung von IoT-Plattformen}
- (Fortsetzung 7.3)\\
- Welche Plattformen eignen sich für welche Anwendungsfälle?\\
- Welche Kriterien sind besonders wichtig für welche Anwendungsfälle?\\
\section{Bewertung des Auswahlprozesses anhand des Kriterienkatalogs}\label{sec:Bewertung des Auswahlprozesses anhand des Kriterienkatalogs}
- (zu 7.2/7.3)\\
- Ist der Kriterienkatalog ausreichend?\\
- Welche Schwierigkeiten ergeben siche bei der Anwendung auf die IoT-Plattformen?\\
\section{Ergebnisse der exemplarischen Umsetzung des Auswahlprozesses}\label{sec:Ergebnisse der exemplarischen Umsetzung des Auswahlprozesses}
-(zu 8)\\

    \input{chapters/10 Zusammenfassung.tex}
    \input{chapters/11 Ausblick.tex}
 
\end{document} 