\chapter{Stand der Technik}\label{ch:Stand der Technik}
\section{Komponenten einer IoT-Lösung}\label{sec:Komponenten einer IoT-Lösung}
% [Guth2018] [10.1007/978-3-030-24311-1_8]\\
\subsection{Geräte}\label{subsec:Geräte}
- Sensoren und Aktoren\\
- Treiber für Sensoren und Aktoren\\
- Geräte SDKs\\
- RTOS\\
- Edge Computing\\
\subsection{Gateway}\label{subsec:Gateway}
- Software und Schnittstellen für Gerätekommunikation\\
- Gateway SDKs\\
- OS\\
- Schnittstellen zu Cloud-Plattformen (MQTT, REST, etc.)\\
- Fog Computing\\
\subsection{IoT-Software-Plattform}\label{subsec:IoT-Software-Plattform}
- Verwaltet Geräte und Gateways\\
- Stellt Schnittstellen für externe Applikationen und Services bereit\\
- bietet Services für die Verarbeitung von Daten\\
- wird in dieser Arbeit als IoT-Plattform bezeichnet\\
\subsection{Anwendung}\label{subsec:Anwendung}
- Anwendungen / Applikationen / Services auf Basis der IoT-Plattform und der Daten der Geräte\\
\section{Referenzarchitektur von IoT-Software-Plattformen}\label{sec:Referenzarchitektur von IoT-Software-Plattformen}
\subsection{Kommunikation}\label{subsec:Kommunikation}
- MQTT Broker\\
- (MQTT-)WebSockets\\
- HTTP (REST)\\
\subsection{Digitaler Zwilling}\label{subsec:Digitaler Zwilling}
- Device Registry / Twin\\
- Device Metadata and Data\\
- Remote Actions\\
- Device Updates\\
\subsection{IoT-Services}\label{subsec:IoT-Services}
% [Lempert2021] [Microsoft_2021] [9825678]\\
- Develop Insights (Cold, Warm and Hot Path)\\
- Databases (Structured, Unstructured and Time Series)\\
- Descriptive Analytics\\
- Diagnostic Analytics\\
- Real-Time Analytics (Stream Processing Hot Path)\\
- Predictive / Prescriptive Analytics (Machine Learning Cold Path)\\
\subsection{Integration in Drittsysteme}\label{subsec:Integration in Drittsysteme}
- Adapter zu Enterprise-Systemen\\
- Cloud to Cloud Integration\\
- Apis?\\
\subsection{Orchestrierung von Services}\label{subsec:Orchestrierung von Services}
- Message Broker / Event Bus\\
- Rules Engine\\
- Service-Orchestrierung\\
- Load Balancing?\\
\subsection{Management}\label{subsec:Management}
- Device Management -> Device Twin\\
- Configuration / State Management -> Device Twin\\
- Errors and Reports\\
\subsection{Security}\label{subsec:Security}
- Authentication / Authorization\\
- Identity Management\\
\section{Aktuelle Implementierungen von IoT-Software-Plattformen}\label{sec:Aktuelle Implementierungen von IoT-Software-Plattformen}
\subsection{Hyperscaler-Cloud-Anbieter}\label{subsec:Hyperscaler-Cloud-Anbieter}
- Bindung an den Cloud-Anbieter?\\
- Regional verteilbar -> geringere Latenz?\\
\subsection{(Open-Source IoT-Plattformen)}\label{subsec:(Open-Source IoT-Plattformen)}
- Cloud agnostisch?\\
- On-Premise möglich?\\
\subsection{(IoT-Spezifische Plattformen)}\label{subsec:(IoT-Spezifische Plattformen)}
% ^ möglicherweise zusammenfassen ^\\
\subsection{Low-Code-Plattformen}\label{subsec:Low-Code-Plattformen}
- schnellere Entwicklung von IoT-Lösungen?\\
- höhere Flexibilität?\\
- hohe Abhängigkeit von der Plattform?\\
- hohe Kosten?\\
\section{IoT Geschäftsmodellentwicklung}\label{sec:IoT Geschäftsmodellentwicklung}
\subsection{Prototypentwicklung}\label{subsec:Prototypentwicklung}
\subsection{Skalierung}\label{subsec:Skalierung}
