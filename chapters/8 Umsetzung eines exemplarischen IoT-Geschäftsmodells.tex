\chapter{Umsetzung eines exemplarischen IoT-Geschäftsmodells}\label{ch:Umsetzung eines exemplarischen IoT-Geschäftsmodells}
\section{Auswahl einer IoT-Plattform für ein exemplarisches Geschäftsmodell}\label{sec:Auswahl einer IoT-Plattform für ein exemplarisches Geschäftsmodell}
\subsection{Auswahl eines exemplarischen Geschäftsmodells (groß)}\label{subsec:Auswahl eines exemplarischen Geschäftsmodells (groß)}
\subsection{Auswahl einer IoT-Plattform anhand des Kriterienkatalogs}\label{subsec:Auswahl einer IoT-Plattform anhand des Kriterienkatalogs}
\subsection{Evaluation des Auswahlprozesses anhand des Geschäftsmodells}\label{subsec:Evaluation des Auswahlprozesses anhand des Geschäftsmodells}
\section{Prototypische Implementierung des Geschäftsmodells}\label{sec:Prototypische Implementierung des Geschäftsmodells}
\subsection{Umsetzung einer Teilfunktionalität auf der IoT-Plattform (klein)}\label{subsec:Umsetzung einer Teilfunktionalität auf der IoT-Plattform (klein)}
\subsection{Simulation des Betriebs der skalierten Lösung}\label{subsec:Simulation des Betriebs der skalierten Lösung}
\subsection{Evaluation der Umsetzung}\label{subsec:Evaluation der Umsetzung}
- optionale Funktionen / extras\\
- Sicherheit\\
- Skalierbarkeit\\
- Kosten wie erwartet?\\
- Kostenentwicklung wie erwartet?\\
- Kostenloser Einstieg?\\
- Developer Experience\\
- Vorkenntnisse wie erwartet?\\
- Bedienbarkeit\\
- Performanz (Global vs. Regional)\\
- Analysen\\
- Anzahl der Geräte\\
- Anzahl der Anwendungen\\
- Kosten für Performanz\\
- Fehler?\\
- Wie verhält sich das System bei Ausfällen?\\
- Bewusstes zerstören -> wie verhält sich das System?\\
- Wie verhält sich das System bei Lastspitzen?\\
- Regeneriert sich das System selbst?\\
