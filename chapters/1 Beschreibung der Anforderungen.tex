\chapter{Beschreibung der Anforderungen}\label{ch:Beschreibung der Anforderungen}
\pagenumbering{arabic}
\section{Das Kühlschrank-Verwaltungssystem}\label{sec:Das Kühlschrank-Verwaltungssystem}

Die im Rahmen dieser Ausarbeitung zu entwerfende Datenbank ist Teil eines Projektes zur Entwicklung eines Verwaltungssystems für Lebensmittel, die in Kühl- oder Gefrierschränken gelagert werden. Ziel des Projektes ist es, eine günstige Alternative zu smarten Kühlschränken zu entwickeln, die es ermöglicht die Lagerung von Lebensmitteln zu überwachen, ohne einen neuen Kühlschrank anschaffen zu müssen. Mit Hilfe der Lösung soll der Verschwendung von Lebensmitteln entgegengewirkt werden.\\ In vielen Haushalten werden aufgrund mangelnder Planung bzw. mangelndem Überblick über die Situation im Kühlschrank teilweise große Mengen an Lebensmitteln entsorgt. Zum Einen geraten Lebensmittel die im Kühlschrank weiter hinten stehen in Vergessenheit und werden nicht rechtzeitig verwendet. Zum Anderen werden Lebensmittel gekauft, ohne dass klar ist, wann und in welcher Form dieses Benötigt werden. Die zu entwickelnde Lösung soll beiden Problemen entgegenwirken. Durch die Überwachung der Lebensmittellagerung wird die Planung von Einkauf und Verwertung dieser erleichtert.\\ Aus Benutzer:innen Sicht stellt eine Smartphone-App das zentrale Element der Lösung dar. Über diese können Kühl- und Gefrierschränke, oder andere Lagerstätten für Lebensmittel eingebunden und verwaltet werden. Auf einen Blick sollen Benutzer:innen alle Lebensmittel in all ihren Lagerstätten aufgelistet bekommen. Zusätzlich soll der Füllstand bzw. die Menge der jeweiligen Lebensmittel und die Haltbarkeit angezeigt werden. Benutzer:innen sollen zusätzlich die Möglichkeit haben detaillierte Informationen wie Zutatenlisten einzelner Produkte ab zu rufen. Mit Hilfe der App können Benutzer:innen darüber hinaus ihren Einkauf planen und nicht automatisierbare Ein- bzw. Auslagerungsprozesse manuell einpflegen.\\ Damit Benutzer:innen die Lösung tatsächlich verwenden muss die Bedienung so komfortabel wie möglich sein. Besonders kritisch ist die Anzahl der manuell zu pflegenden Prozesse. Lebensmittel sollen mit Hilfe eines IoT-Gerätes automatisch Ein- bzw. Ausgelagert werden können. Der Großteil an Produkten wird mit Hilfe eines Barcode-Scanners erfasst. Produkte ohne Barcode sollen mittels bildgestützter Verfahren erkannt werden. Um die IoT-Geräte optimal einbinden und verwalten zu können sollte die Lösung eine IoT-Plattform integrieren.\\ Bei der Entwicklung der Lösung werden spätere potentielle Erweiterungen mit bedacht. Unter dem Aspekt der Lebensmittelverschwendung wäre die Integration einer Rezepteverwaltung bzw. einer Ernährungsplanung besonders nützlich. Auch die Analyse von Nutzungsdaten kann potentiell dazu Verwendet werden die Zahl der nötigen manuellen Eingriffe von Benutzer:innen zu reduzieren. Abseits der Anwendungsfälle des Lebensmittelverwaltungssystems, kann eine doppelte Nutzung der in Kühlschränken installierten IoT-Geräte sinnvoll sein. So könnten zum Beispiel Temperaturdaten von Kühl- und Gefrierschränken interessant für Projekte sein, die den Energieverbrauch von Haushalten optimieren.\\ Die im Rahmen dieser Ausarbeitung zu entwickelnde Lösung stellt das \glsxtrfull{MVP}, also das minimal funktionsfähige Produkt des gesamten Verwaltungssystems dar. Der Fokus liegt dabei auf dem Entwurf der Datenbank, sowie den zum Test der Datenbank benötigten Komponenten. Es wird Prototyp für das \gls{Backend} in Node.js entwickelt, dass an die Datenbank angebunden ist. Dieses enthält die nötige Verwaltungslogik des Systems und bietet eine \gls{REST-API} für die Anbindung der Smartphone-App. Über die App wird es in der \glsxtrshort{MVP}-Phase nur möglich sein die Liste der Aktuell eingelagerten Lebensmittel an zu zeigen.

\section{Anforderungen an die Datenbank}\label{sec:Anforderungen an die Datenbank}

Aus dem beschriebenen Projektziel ergeben sich die Anforderungen an die zu entwerfende Datenbank. Da die Daten unterschiedlicher Benutzer:innen durch ein Zentrales \gls{Backend} verarbeitet werden wird eine Benutzer:innenverwaltung benötigt. Die Datenbank muss also die Möglichkeit bieten Benutzer:innen inklusive ihrer Anmeldedaten zu speichern. Darüber hinaus müssen allen Benutzer:innen ihre Geräte (also Kühl- oder Gefrierschränke, etc.) sowie ihre Einkaufslisten zugeordnet werden. Alle nicht einzelnen Benutzer:innen zuordbaren \Gls{Entitätsklasse}n sollen möglichst übergreifend genutzt werden.\\ Neben den Benutzer:innen müssen auch die Geräte der Benutzer:innen verwaltet und entsprechend in der Datenbank gespeichert werden. Jedes der in der Datenbank gespeicherten Geräte repräsentiert dabei ein real existierendes IoT-Gerät, welches an einer Lagerstätte für Lebensmittel installiert ist. Das Gerät stellt damit auch die Schnittstelle für ein an zu knüpfende IoT-Plattform dar.\\ Damit Benutzer:innen in der Lage sind ihren Lebensmittelhaushalt zu überwachen, müssen die Lebensmittel, die eine Lagerstätte enthält gespeichert werden. Benutzer:innen sollen zugriff auf eine Reihe von Informationen über die Lebensmittel in ihren Kühlschränken haben. Es müssen also die Menge, das Öffnungs- bzw. Kaufdatum und die \gls{EAN} aller Produkte gespeichert werden. Damit Produkte aus dem Kühlschrank herausgenommen und später wieder hineingelegt werden können, ist es außerdem sinnvoll eine Produkthistorie zu speichern. Zur besseren Übersicht der Benutzer:innen sollen Produkte außerdem Kategorisiert werden können. Auch diese Eigenschaft der Produkte muss durch das Datenbankenmodell abbildbar sein. Die Einordnung von Produkten in Kategorien ist darüber hinaus für weitere optionale Funktionen, wie die Einbindung von Rezepten hilfreich.\\ Damit für alle Benutzer:innen das Erscheinungsbild von Anwendungen die auf die Datenbank zurückgreifen gleichartig ist, sollen auch Anzeigeelemente, die spezifischen Entitäten zugeordnet sind in der Datenbank gespeichert werden. So lässt sich sicherstellen, dass zum Beispiel das Icon einer Produktkategorie überall identisch ist und nicht in unterschiedlichen Versionen oder Anwendungen variiert. Auch wird so verhindert, dass bei der Einführung neuer Kategorien die Anwendungsprogramme angepasst werden müssen.\\ Neben der Speicherung des Ist-Zustandes soll aus diesem außerdem automatisiert eine Einkaufsliste erstellt werden können. Produkte, welche immer vorhanden sein sollen, müssen automatisch auf eine Einkaufsliste gesetzt werden, welche durch den Benutzer bearbeitet werden kann. Perspektivisch ist auch die automatisierte Verarbeitung von Rezepten zu einer Einkaufsliste denkbar und sollte durch das Datenbankmodell abbildbar sein.\\ Da bereits bekannt ist, dass es sich bei der zu entwickelnden Lösung um ein \glsxtrshort{MVP} handelt, ist die Datenbank wann immer möglich so zu entwerfen, dass die bereits geplanten späteren Funktionen auf einfache weise integriert werden können. So soll verhindert werden, dass die Daten aus der Datenbank zu einem späteren Zeitpunkt aufwändig migriert werden müssen. 

\section{Abgrenzung der Anforderungen}\label{sec:Abgrenzung der Anforderungen}

Wie bereits beschrieben, soll im Rahmen dieser Ausarbeitung lediglich ein \glsxtrshort{MVP} des gesamten Projekts entwickelt werden. Es ist also nicht gefordert bereits alle Anforderungen an die fertige Lösung zu erfüllen. Viele Funktionen werden in dieser ersten Entwicklungsiteration zunächst nur simuliert oder vollständig ausgelassen. Die zu entwickelnde Smartphone-App dient vor allem zum anschaulichen Test der zu entwickelnden \glsxtrshort{API} und ist noch nicht als fertiges Anwendungsprogramm zu sehen. Ein großteil der späteren Funktionalität wird noch nicht implementiert und auch das Benutzer:inneninterface hat noch nicht den Anspruch intuitiv oder ansprechend zu sein. Für Tests der Gesamten Lösung ist eine prototypische Implementierung aber durchaus geeignet.\\ Die Anbindung von Geräten wird in der ersten Iteration vollständig simuliert. Da noch keine Geräte entwickelt werden, ist auch die Anbindung einer IoT-Plattform noch wenig sinnvoll. Die in dieser Phase bereits zu entwickelnde Schnittstelle für die Gerätekommunikation im Backend wird durch ein Python-Skript angesprochen. Mit diesem Skript soll das Herausnehmen und das Hineinlegen von Produkten mit einer bestimmten \glsxtrshort{EAN} simuliert werden. Das Anlegen neuer Geräte kann ohne die Verwendung einer IoT-Plattform noch nicht realitätsgetreu durchgeführt werden. Auch diese Funktion wird durch ein Python-Skript und eine zu entwickelnde \glsxtrshort{API} simuliert.\\ Eine spätere hoch skalierbare Lösung soll bei einem Cloud-Dienstleister betrieben werden. Da sich die Lösung noch in der Prototypphase befindet und noch von keinen Benutzer:innen verwendet wird, sollen die Datenbank und das Backend zunächst lokal auf dem Rechner des Entwicklers betrieben werden. Die Lösung soll vollständig in Entwicklungscontainern ausführbar sein, um komfortabel Tests durchführen zu können.\\ Auf verschiedene Sicherheitsmaßnahmen kann in dieser Phase verzichtet werden, da noch keine Daten echter Benutzer:innen verarbeitet werden und die ein unerlaubter Zugriff auf die \glsxtrshort{API} oder die Datenbank in diesem Stadium noch nicht kritisch ist. Die sowohl das Backend, als auch die Datenbank werden vom Entwickler lokal betrieben und sind somit lediglich potentiellen Angreifern im eigenen Netzwerk ausgeliefert. Da der Wert der Daten in der Datenbank zu diesem Zeitpunkt gering ist das Risiko vertretbar. Benutzer:innen oder Anwendungsprogramme sollen sich für die Benutzung der \glsxtrshort{API} weder Autorisieren noch Authentifizieren müssen.