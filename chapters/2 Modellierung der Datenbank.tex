\chapter{Modellierung der Datenbank}\label{ch:Modellierung der Datenbank}
\section{Konzeptionelles Schema}\label{sec:Konzeptionelles Schema}

Aus den in Kapitel~\ref{sec:Anforderungen an die Datenbank} beschriebenen Anforderungen lassen sich bereits verschiedene Entitätsklassen und deren Attribute ableiten. Das Resultierende Konzeptionelle Schema ist in Abbildung~\ref{fig:2.1} dargestellt. Für die Verwaltung von Benutzer:innen wird die Entitätsklasse \textit{fridge\_user} verwendet. An diese wird zunächst nur die Anforderung gestellt, dass sich Benutzer:innen anmelden können. Um das zu gewährleisten werden der \textit{login} von Benutzer:innen sowie ein \Gls{Hash} des Passwortes gespeichert.\\ Alle Benutzer:innen sollen die Möglichkeit haben mehrere Geräte zu überwachen. Gleichzeitig kann zum Beispiel in einer Wohngemeinschaft ein Gerät auch mehreren Benutzer:innen gehören bzw. von diesen Überwacht werden. Es wird also eine Entitätsklasse \textit{device} benötigt, die eine \Gls{MM} zu \textit{fridge\_user} hat. Ein Gerät soll neben einer eindeutigen ID zur Identifikation des Gerätes auch einen Namen haben, den Benutzer:innen vergeben können, um ihre Geräte auseinander halten zu können.\\ Damit Benutzer:innen auch Produkte in ihren Kühlschränken lagern können, wird eine weitere Entitätsklasse \textit{product} benötigt. Zwischen \textit{device} und \textit{product} gibt es eine \Gls{MM}, da unterschiedliche Kühlschränke jeweils die gleichen Produkte enthalten können, ein Gerät aber auch mehrer verschiedene Produkte enthalten kann. Produkte repräsentieren dabei sowohl Lebensmittel mit Barcode, als auch solche ohne. Bei Produkten mit Barcode können die wesentlichen Informationen zu dem jeweiligen Produkt mit der \glsxtrshort{EAN} von einer externen \gls{API} abgerufen werden. Diese Informationen sind in Abbildung~\ref{fig:2.1} als Attribute mit gestrichelter Linie dargestellt. Neben den produktspezifischen Informationen sind für das Lebensmittelverwaltungssystem zusätzliche Informationen interessant. So muss zum Beispiel bekannt sein, wie lange ein Produkt bereits gelagert wird bzw. wie lange es bereits geöffnet ist. Auch der Füllstand eines Produktes muss gespeichert werden. Um der Anforderung einer Produkthistorie gerecht zu werden, muss auch der Zeitpunkt der Entnahme eines Produktes gespeichert werden.
\fig{2.1}{Konzeptionelles Schema}
Für Produkte ohne Barcode müssen auch die produktspezifischen Informationen in der Datenbank abbildbar sein. Dazu wird die Entitätsklasse \textit{product\_class} verwendet. Jedes Produkt ist bis zu einer Produktklasse zu zu ordnen, es gibt als eine \gls{1M}. Auch Produkte mit Barcode können einer Produktklasse zugeordnet werden, so kann zum Beispiel das Produkt mit der \glsxtrshort{EAN} \textit{4004980806405} der Klasse \textit{Erdnüsse} zugeordnet werden. Produktklassen haben neben einer eindeutigen ID die Attribute \textit{class\_name}, \textit{class\_image} und \textit{unit\_symbol}. Darüber hinaus lassen sich Produktklassen einer Produktkategorie zuordnen. Erdnüsse würden zum Beispiel der Kategorie Nüsse angehören.\\ Damit Benutzer:innen Einkaufslisten erstellen können, gibt es zusätzliche die Tabelle \textit{shopping\_list}. Diese kann von Benutzer:innen manuell bearbeitet werden. Jedem Gerät ist zusätzlich eine Tabelle \textit{required\_content} zugeordnet. Fragen Benutzer:innen ihre Einkaufsliste ab, werden alle Produkte von \textit{shopping\_list}, sowie alle von \textit{required\_content}, welche nicht im Kühlschrank vorrätig sind gelistet.

\section{Physisches Schema}\label{sec:Physisches Schema}

-

\fig{2.2}{Physisches Schema}

\section{Nachweis der Normalformen}\label{sec:Nachweis der Normalformen}
