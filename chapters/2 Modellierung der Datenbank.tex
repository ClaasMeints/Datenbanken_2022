\chapter{Modellierung der Datenbank}\label{ch:Modellierung der Datenbank}
\section{Konzeptionelles Schema}\label{sec:Konzeptionelles Schema}

Aus den in Kapitel~\ref{sec:Anforderungen an die Datenbank} beschriebenen Anforderungen lassen sich bereits verschiedene Entitätsklassen und deren Attribute ableiten. Das Resultierende Konzeptionelle Schema ist in Abbildung~\ref{fig:2.1} dargestellt. Für die Verwaltung von Benutzer:innen wird die Entitätsklasse \textit{fridge\_user} verwendet. An diese wird zunächst nur die Anforderung gestellt, dass sich Benutzer:innen anmelden können. Um das zu gewährleisten werden der \textit{login} von Benutzer:innen sowie ein \Gls{Hash} des Passwortes gespeichert.\\ Alle Benutzer:innen sollen die Möglichkeit haben mehrere Geräte zu überwachen. Gleichzeitig kann zum Beispiel in einer Wohngemeinschaft ein Gerät auch mehreren Benutzer:innen gehören bzw. von diesen Überwacht werden. Es wird also eine Entitätsklasse \textit{device} benötigt, die eine \Gls{MM} zu \textit{fridge\_user} hat. Ein Gerät soll neben einer eindeutigen ID zur Identifikation des Gerätes auch einen Namen haben, den Benutzer:innen vergeben können, um ihre Geräte auseinander halten zu können.\\ Damit Benutzer:innen auch Produkte in ihren Kühlschränken lagern können, wird eine weitere Entitätsklasse \textit{product} benötigt. Zwischen \textit{device} und \textit{product} gibt es eine \Gls{MM}, da unterschiedliche Kühlschränke jeweils die gleichen Produkte enthalten können, ein Gerät aber auch mehrer verschiedene Produkte enthalten kann. Produkte repräsentieren dabei sowohl Lebensmittel mit Barcode, als auch solche ohne. Bei Produkten mit Barcode können die wesentlichen Informationen zu dem jeweiligen Produkt mit der \glsxtrshort{EAN} von einer externen \gls{API} abgerufen werden. Diese Informationen sind in Abbildung~\ref{fig:2.1} als Attribute mit gestrichelter Linie dargestellt. Neben den produktspezifischen Informationen sind für das Lebensmittelverwaltungssystem zusätzliche Informationen interessant. So muss zum Beispiel bekannt sein, wie lange ein Produkt bereits gelagert wird bzw. wie lange es bereits geöffnet ist. Auch der Füllstand eines Produktes muss gespeichert werden. Um der Anforderung einer Produkthistorie gerecht zu werden, muss auch der Zeitpunkt der Entnahme eines Produktes gespeichert werden.
\fig{2.1}{Konzeptionelles Schema}{1}
Für Produkte ohne Barcode müssen auch die produktspezifischen Informationen in der Datenbank abbildbar sein. Dazu wird die Entitätsklasse \textit{product\_class} verwendet. Jedes Produkt ist bis zu einer Produktklasse zu zu ordnen, es gibt als eine \gls{1M}. Auch Produkte mit Barcode können einer Produktklasse zugeordnet werden, so kann zum Beispiel das Produkt mit der \glsxtrshort{EAN} \textit{4004980806405} der Klasse \textit{Erdnüsse} zugeordnet werden. Produktklassen haben neben einer eindeutigen ID die Attribute \textit{class\_name}, \textit{class\_image} und \textit{unit\_symbol}. Darüber hinaus lassen sich Produktklassen einer Produktkategorie zuordnen. Erdnüsse würden zum Beispiel der Kategorie Nüsse angehören.\\ Damit Benutzer:innen Einkaufslisten erstellen können, gibt es zusätzliche die Tabelle \textit{shopping\_list}. Diese kann von Benutzer:innen manuell bearbeitet werden. Jedem Gerät ist zusätzlich eine Tabelle \textit{required\_content} zugeordnet. Fragen Benutzer:innen ihre Einkaufsliste ab, werden alle Produkte von \textit{shopping\_list}, sowie alle von \textit{required\_content}, welche nicht im Kühlschrank vorrätig sind gelistet.

\section{Physisches Schema}\label{sec:Physisches Schema}

Im nächsten Schritt wird aus dem Konzeptionellen Schema das Physische Schema erstellt. Dieses ist in Abbildung~\ref{fig:2.2} dargestellt. Während die Tabelle \textit{fridge\_user} unverändert übernommen werden kann, muss die \gls{MM} zum Gerät aufgelöst werden. Es wird die zusätzliche Tabelle \textit{fridge\_user\_device\_relation} eingeführt, welche die Geräte den Benutzer:innen zuordnet. Gemäß dem Standardvorgehen für die Auflösung von \gls{MM}en werden die Primärschlüssel von \textit{fridge\_user} und \textit{device} als Fremdschlüssel der zusammengesetzte Primärschlüssel von \textit{fridge\_user\_device\_relation}. Wie auch \textit{fridge\_user} kann die Tabelle \textit{device} unverändert übernommen werden.\\ Bei der Auflösung der \gls{MM} zwischen \textit{device} und \textit{product} wechseln einige Attribute von \textit{product} zu der neu geschaffenen Relationstabelle \textit{device\_content}. Während Produkte in diesem Kontext abstrakte Beschreibungen eines Produkttyps repräsentieren, sind die Entitäten der Tabelle \textit{device\_content} als tatsächliche Lebensmittel in einem Kühlschrank zu verstehen. Entsprechend sind Informationen, wie Einlagerungs-, Öffnungs- und Auslagerungsdatum, sowie der Füllstand in \textit{device\_content} geführt. Der Primärschlüssel wird aus dem Fremdschlüssel des Gerätes und dem Einlagerungsdatum gebildet, da jedes Gerät zeitgleich nur ein Produkt erfassen kann.
\fig{2.2}{Physisches Schema}{1}
Die \gls{1M} zwischen \textit{product} und \textit{product\_class} kann unverändert bestehen bleiben. Der Tabelle \textit{product\_class} wird ein zusätzliches Attribut \textit{storage\_life\_opened} hinzugefügt. Dies ist notwendig, da diese Information sonst bei Produkten ohne Barcode nicht verfügbar wäre. Darüber hinaus werden die beiden Attribute \textit{product\_category} und \textit{unit(\_symbol)} in eingene Tabellen umgewandelt. So wird sichergestellt, dass nur zuvor festgelegte Werte angenommen werden können. Außerdem wird sichergestellt, dass Produkte nach Einheit und Kategorie sortiert werden können. Auch zusätzliche Informationen, wie der Name einer Einheit oder ein Icon für die Darstellung einer Kategorie in Anwendungsprogrammen sind so abbildbar. Neben der Tabelle \textit{unit} gibt es eine zusätzliche Einheit \textit{unit\_conversion}. Diese bildet Informationen über die Umrechnung von Einheiten ineinander ab. Möchten Benutzer:innen zum Beispiel stets 2Kg Erdnüsse vorrätig haben, hat aber eine 200g Dose Erdnüsse im Regal, kann mit Hilfe dieser Informationen dennoch die fehlende Menge an Erdnüssen bestimmt werden.\\ Bei der Überführung der Tabellen \textit{required\_content} und \textit{shopping\_list} ergeben sich weitere Veränderungen. Beide Tabellen werden zu Relationstabellen, welche jeweils Produktklassen entweder einem Gerät oder einem:einer Benutzer:in zuordnet. Die Tabellen enthalten jeweils die beiden Fremdschlüssel als zusammengesetzten Primärschlüssel sowie die Menge der jeweiligen Produktklasse, die benötigt wird bzw. eingekauft werden soll.

\section{Nachweis der Normalformen}\label{sec:Nachweis der Normalformen}

Um eine Reihe von häufig auftretenden Modellierungsfehlern wie Redundanzen zu vermeiden und Anomalien im Betrieb zu verhindern wird im nächsten Schritt das Datenbankmodell normalisiert. Im Zuge dieser Normalisierung wird das Modell auf die Einhaltung der ersten drei Normalformen hin überprüft und gegebenenfalls angepasst.\\ Um die Einhaltung der \glsxtrfull{1NF} zu prüfen, müssen die Attribute aller Tabellen betrachtet werden. Es ist zu prüfen, ob diese atomar sind, also sich nicht weiter zerlegen lassen. In Tabelle~\ref{tab:2.1} sind alles Attribute, mit Ausnahme von Fremdschlüsseln dargestellt. Für alle Attribute außer den beiden Bilder \textit{class\_image} und \textit{category\_icon} ist anhand der Beispiele ersichtlich, dass diese atomar sind. Bei den Bilddaten handelt es sich um listenwertige Attribute, da diese aber nicht separat voneinander verwendet werden sondern ein einzelnes Bild repräsentieren, liegt dennoch kein verstoß gegen die \glsxtrshort{1NF} vor. Da für keines der Attribute des Schemas ein Verstoß gegen die \glsxtrshort{1NF} vorliegt, erfüllen auch alle Tabellen die \glsxtrshort{1NF}.\\ Für den Nachweis der \glsxtrfull{2NF} darf kein Attribut einer Tabelle bereits von einem Teil des zusammengesetzten Primärschlüssels abhängig sein. Die \glsxtrshort{2NF} ist also für alle Tabellen mit einfachem Primärschlüssel trivialerweise erfüllt. Damit ist der Nachweis für \textit{fridge\_user\_device\_relation}, \textit{device\_content}, \textit{unit\_conversion}, \textit{required\_content} und \textit{shopping\_list} zu erbringen. Da \textit{fridge\_user\_device\_relation} außer dem zusammengesetzten Primärschlüssel keine weiteren Attribute enthält, ist auch hier die \glsxtrshort{2NF} trivialerweise erfüllt.
\begin{table}
    \centering
    \begin{tabular}{l|l|l|c}
        Tabelle & Attributname & Beispielwert & atomar \\
        \hline
        fridge\_user      & login                 & "claas"                        & \ding{51} \\
        fridge\_user      & password              & "\$2a\$12\$C16E0aHgerPbr..."   & \ding{51} \\
        device            & device\_id            & 1                              & \ding{51} \\
        device            & device\_name          & "Kühlschrank"                  & \ding{51} \\
        device\_content   & filled\_in            & 2022-09-23 12:22:03.142+00     & \ding{51} \\
        device\_content   & opened                & 2022-09-23 12:22:03.142+00     & \ding{51} \\
        device\_content   & dropped\_out          & 2022-09-23 12:22:03.142+00     & \ding{51} \\
        device\_content   & percentage\_left      & 100                            & \ding{51} \\
        product           & product\_id           & 1                              & \ding{51} \\
        product           & ean                   & 4004980806405                  & \ding{51} \\
        product\_class    & class\_id             & 1                              & \ding{51} \\
        product\_class    & class\_name           & ''Erdnüsse"                    & \ding{51} \\
        product\_class    & class\_image          & "253, 80, 78, 10, 40, 43, ..." & \ding{55} \\
        product\_class    & storage\_life\_opened & 7                              & \ding{51} \\
        product\_category & category\_id          & 1                              & \ding{51} \\
        product\_category & category\_name        & "Nüsse"                        & \ding{51} \\
        product\_category & category\_icon        & "253, 80, 78, 10, 40, 43, ..." & \ding{55} \\
        unit              & unit\_id              & 1                              & \ding{51} \\
        unit              & unit\_symbol          & "g"                            & \ding{51} \\
        unit              & unit\_name            & "Gramm"                        & \ding{51} \\
        unit\_conversion  & factor                & 1000                           & \ding{51} \\
        required\_content & quantity              & 1                              & \ding{51} \\
        shopping\_list    & quantity              & 1                              & \ding{51} \\
    \end{tabular}
    \caption{Erste Normalform der Attribute}
    \label{tab:2.1}
\end{table}
Die Tabelle \textit{device\_content} enthält neben dem Primärschlüssel Eigenschaften, die einem Konkreten Lebensmittel zugeordnet werden. Diese sind nicht von der \textit{device\_id} abhängig, da ein Gerät unterschiedliche Lebensmittel enthalten kann. Auch von \textit{filled\_in}-Datum sind diese Attribute unabhängig, da in unterschiedliche Geräten unterschiedliche Lebensmittel zum gleichen Zeitpunkt (gleiche Millisekunde) hineingelegt werden können.\\ Auch für \textit{unit\_conversion} ist die \glsxtrshort{2NF} erfüllt, da Umrechnungsfaktor zwischen zwei Einheiten nicht nur von einer der Einheiten abhängig sein kann. Für \textit{required\_content} und \textit{shopping\_list} ist jeweils die Menge von der Produktklasse unabhängig, da unterschiedliche Geräte bzw. Benutzer:innen unterschiedlich viel einer Produktklasse voraussetzen bzw. einkaufen können. Auch umgekehrt gilt, dass Benutzer:innen bzw. Geräte von unterschiedlichen Produktklassen unterschiedliche Mengen einkaufen bzw. voraussetzen können, die \textit{quantity} als voll vom zusammengesetzten Primärschlüssel abhängt. Die \glsxtrshort{2NF} ist also für alle Tabellen erfüllt.\\ Damit auch die \glsxtrfull{3NF} erfüllt ist, müssen zusätzlich zur \glsxtrshort{2NF} alle Determinanten Schlüsselkandidaten sein. Für alle Tabellen in denen es keine Abhängigkeiten von Attributen gibt, die nicht Teil des Primärschlüssels sind, ist die \glsxtrshort{3NF} trivialerweise erfüllt.\\ Im Fall der Tabelle \textit{product} sind viel der im Konzeptionellen Schema enthaltenen Attribut von der \glsxtrshort{EAN} abhängig. Da diese kein Primärschlüsselkandidat ist, wurden die Attribute aus dem Physischen Schema entfernt. Die Daten können allerdings durch eine externe \glsxtrshort{API} abgerufen werden und müssen deshalb nicht in einer separaten Tabelle vorgehalten werden.\\ Bei den Tabellen \textit{product\_class}, \textit{product\_category} und \textit{unit} gäbe es jeweils eine Abhängigkeit anderer Attribute vom Namen, falls dieser eindeutig ist. Da in jedem Fall mehrere Entitäten mit dem gleichen Namen ausdrücklich erlaubt sind, ist dies nicht der Fall.\\ In der Tabelle \textit{unit\_conversion} ist jede Kombination aus zwei Attributen Primärschlüsselkandidat, es kann also keine Determinanten geben, die keine Primärschlüsselkandidaten sind. Die \glsxtrshort{3NF} ist also für alle Tabellen erfüllt und die Normalisierung des Datenbankenmodells ist vollständig.

\section{Domänen und Standardwerte}\label{sec:Domänen und Standardwerte}

Im Letzten Schritt der Model der Datenbank werden die Domänen und weitere Eigenschaften der Attribute festgelegt. In Tabelle~\ref{tab:2.2} sind für alle Attribute ausgenommen der Fremdschlüssel die Domänen dargestellt. Außerdem ist angegeben, ob diese den Wert \textit{Null} haben dürfen und ob diese automatisch Inkrementiert werden. Einige der Attribute werden zusätzlich mit einem Standardwert versehen, welcher immer verwendet wird, wenn kein Wert für das Attribut angegeben ist. Für alle Attribut mit Standardwert sind diese in Tabelle~\ref{tab:2.3} dargestellt. Da es zwei Attribute mit dem Namen \textit{quantity} gibt sind diese jeweils mit einer Nummer versehen. Bei (1) handelt es sich um das Attribut aus der Tabelle \textit{required\_content}, bei (2) um das aus \textit{shopping\_list}.

\begin{table}
    \centering
    \begin{tabular}{l|l|c|c}
        Attribut & Domäne & Nullbar & Auto Inkrement \\
        \hline
        login                 &  CHARACTER VARYING(255)  & \ding{55} & \ding{55} \\
        password              &  CHARACTER VARYING(255)  & \ding{55} & \ding{55} \\
        device\_id            &  INTEGER                 & \ding{55} & \ding{51} \\
        device\_name          &  CHARACTER VARYING(255)  & \ding{55} & \ding{55} \\
        filled\_in            &  TIMESTAMP WITH TIMEZONE & \ding{55} & \ding{55} \\
        opened                &  TIMESTAMP WITH TIMEZONE & \ding{51} & \ding{55} \\
        dropped\_out          &  TIMESTAMP WITH TIMEZONE & \ding{51} & \ding{55} \\
        percentage\_left      &  INTEGER                 & \ding{55} & \ding{55} \\
        product\_id           &  INTEGER                 & \ding{55} & \ding{51} \\
        ean                   &  CHARACTER VARYING(255)  & \ding{51} & \ding{55} \\
        class\_id             &  INTEGER                 & \ding{55} & \ding{51} \\
        class\_name           &  CHARACTER VARYING(255)  & \ding{55} & \ding{55} \\
        class\_image          &  BYTEA                   & \ding{51} & \ding{55} \\
        storage\_life\_opened &  INTEGER                 & \ding{51} & \ding{55} \\
        category\_id          &  INTEGER                 & \ding{55} & \ding{51} \\
        category\_name        &  CHARACTER VARYING(255)  & \ding{55} & \ding{55} \\
        category\_icon        &  BYTEA                   & \ding{51} & \ding{55} \\
        unit\_id              &  INTEGER                 & \ding{55} & \ding{51} \\
        unit\_symbol          &  CHARACTER VARYING(255)  & \ding{55} & \ding{55} \\
        unit\_name            &  CHARACTER VARYING(255)  & \ding{51} & \ding{55} \\
        factor                &  NUMERIC                 & \ding{55} & \ding{55} \\
        quantity(1)           &  NUMERIC                 & \ding{55} & \ding{55} \\
        quantity(2)           &  NUMERIC                 & \ding{55} & \ding{55} \\
    \end{tabular}
    \caption{Domänen der Attribute}
    \label{tab:2.2}
\end{table}

\begin{table}
    \centering
    \begin{tabular}{l|l}
        Attribut & Standardwert \\
        \hline
        device\_name          & "Unbenanntes Gerät" \\
        filled\_in            & CURRENT\_TIMESTAMP  \\
        percentage\_left      & 100                 \\
        factor                & 1                   \\
        quantity(1)           & 1                   \\
        quantity(2)           & 1                   \\
    \end{tabular}
    \caption{Standardwerte der Attribute}
    \label{tab:2.3}
\end{table}
