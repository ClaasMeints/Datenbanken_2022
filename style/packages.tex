% Zur Richtigen Verwendung von Einheiten
\usepackage[
    locale=DE, 
    %per=symbol-or-fraction
]{siunitx}

\sisetup{
    mode = text,
    detect-family,
    detect-weight,  
    exponent-product = \cdot,
    output-decimal-marker={\text{.}},
    %math-rm=\mathsf,
    %text-rm=\sffamily,
    range-phrase = { -- },
    per-mode=symbol,
    per-symbol=/
}
\DeclareSIUnit{\var}{var}

% Anpassung des Seitenlayouts -------------------------------------------------------------% 	siehe Seitenstil.tex
% -------------------------------------------------------------
\usepackage[headsepline,automark]{scrlayer-scrpage}

\usepackage{listings}
\usepackage{xcolor}
% Overall Colors

% VS Code Light+ Theme
% Background: #FFFFFF
\definecolor{background}{HTML}{FFFFFF}
% Foreground: #000000
\definecolor{foreground}{HTML}{000000}
% Comment: #008000
\definecolor{comment}{HTML}{008000}
% String Literal: #A31515
\definecolor{string}{HTML}{A31515}
% Number Literal: #098658
\definecolor{number}{HTML}{098658}
% Keyword: #0000FF
\definecolor{keyword}{HTML}{0000FF}
% Function: #795E26
\definecolor{function}{HTML}{795E26}
% Type: #267F99
\definecolor{type}{HTML}{267F99}
% Control Keyword: #AF00DB
\definecolor{control}{HTML}{AF00DB}
% Decorator: #795E26
\definecolor{decorator}{HTML}{795E26}
% Variable: #001080
\definecolor{variable}{HTML}{001080}

% Define TypeScript Language
\lstdefinelanguage{typescript}{
    backgroundcolor=\color{background},
    keywords={abstract, as, assert, async, class, const, debugger, declare, enum, extends, false, function, get, implements, in, infer, instanceof, interface, is, keyof, let, module, namespace, never, new, null, of, package, private, protected, public, readonly, require, return, set, static, super, symbol, this, true, type, typeof, undefined, unique, unknown, var},
    keywordstyle=\color{keyword},
    comment=[l]{//},
    morecomment=[s]{/*}{*/},
    commentstyle=\color{comment},
    morestring=[b]",
    morestring=[b]',
    stringstyle=\color{string},
    identifierstyle=\color{variable},
    classoffset=1,
    morekeywords={await, break, case, catch, continue, default, delete, do, else, export, finally, for, from, if, import, switch, throw, try, while, with, yield},
    keywordstyle=\color{control},
    classoffset=2,
    morekeywords={any, bigint, boolean, never, number, object, string, symbol, unknown, void, Model, fridge_user},
    keywordstyle=\color{type},
    morecomment=[l]{=>},
    commentstyle=\color{type},
    classoffset=3,
    morekeywords={@Table, @PrimaryKey, @Column, @BeforeCreate, @BeforeUpdate, @HasMany, hash},
    keywordstyle=\color{decorator},
    classoffset=4,  
    morekeywords={ 0, 1, 2, 3, 4, 5, 6, 7, 8, 9},
    keywordstyle=\color{number},
    classoffset=0,
}

% Listing Style
\lstdefinestyle{vscode-light+}{
    basicstyle=\ttfamily\footnotesize,
    breakatwhitespace=false,
    breaklines=true,
    captionpos=b,
    commentstyle=\color{foreground},
    deletekeywords={...},
    escapeinside={\%*}{*)},
    extendedchars=true,
    frame=single,
    keepspaces=true,
    numbers=left,
    numbersep=5pt,
    numberstyle=\color{foreground},
    language=typescript,
    tabsize=2,
}

% Set Listing Style
\lstset{style=vscode-light+, language=typescript}


\usepackage{booktabs}
\usepackage{eurosym}
\usepackage{pifont}% http://ctan.org/pkg/pifont

\renewcommand{\lstlistingname}{Code} % Damit Codebeispiele mit Code betitelt werden.


% Anpassung an Landessprache -------------------------------------------------------------% 	Verwendet globale Option german siehe \documentclass
% -------------------------------------------------------------
\usepackage[ngerman]{babel}
\usepackage{scrhack}
\usepackage[utf8]{inputenc}
\usepackage{amsmath,amssymb,units}
\usepackage{wrapfig,caption, placeins}
\usepackage{multicol}
\captionsetup{
    format = plain,
    justification = centering,
    labelsep = newline,
    singlelinecheck = false,
    labelfont = bf,
    font = small
}
\usepackage{mathptmx, times, helvet, courier} % Pakete für Schriftarten
%\usepackage{mathptmx,charter,helvet,courier}
\usepackage[printonlyused, smaller]{acronym}
\usepackage{wrapfig}
\usepackage{pgfplots}
% Umlaute -------------------------------------------------------------% 		Umlaute/Sonderzeichen wie äöüß direkt im Quelltext verwenden (CodePage).
%		Erlaubt automatische Trennung von Worten mit Umlauten.
% -------------------------------------------------------------
\usepackage[utf8]{inputenc}
\usepackage[T1]{fontenc}
\usepackage{ae}			 	% "schöneres" ä
\usepackage{textcomp} 		% Euro-Zeichen etc.
\usepackage[german=quotes]{csquotes}


% Verwendung von Blindtext zur Layout gestaltung -------------------------------------------------------------
\usepackage{blindtext}

% Grafiken -------------------------------------------------------------
% 		Einbinden von Grafiken [draft oder final]
% 		Option [draft] bindet Bilder nicht ein - auch globale Option
% -------------------------------------------------------------
\usepackage{graphicx}
%\graphicspath{{Grafiken/}} 	% Dort liegen die Bilder des Dokuments


% Befehle aus AMSTeX für mathematische Symbole z.B. \boldsymbol \mathbb ----
\usepackage{amsmath,amsfonts}
\usepackage{tensor}

% Für Index-Ausgabe; \printindex -------------------------------------------------------------
\usepackage{makeidx}

% Einfache Definition der Zeilenabstände und Seitenränder -------------------------------------------------------------
\usepackage{setspace}
\usepackage{geometry}


%% Zum Umfließen von Bildern -------------------------------------------------------------
\usepackage{floatflt}


% Lange URLs umbrechen etc. -------------------------------------------------------------
\usepackage{url}

% für lange Tabellen
\usepackage{longtable}
\usepackage{multirow}
\usepackage{rotating}
\usepackage{diagbox}
\usepackage{array}
\newcolumntype{M}[1]{>{\centering\arraybackslash}m{#1}}
\usepackage{ragged2e}
\usepackage{lscape}
\usepackage{colortbl} %farbige hinterlegung


% Spaltendefinition rechtsbündig mit definierter Breite auf dem Deckblatt 
%------------------------------------------------------------
\newcolumntype{d}[1]{>{\raggedleft\hspace{0pt}}p{#1}}

% Formatierung von Listen ändern
\usepackage{paralist}
% Standardeinstellungen:
\setdefaultleftmargin{2.5em}{2.2em}{1.87em}{1.7em}{1em}{1em}

%Zur Verwendung von BibLaTex
\usepackage[
    style=alphabetic
]{biblatex}
\addbibresource{bibliografie.bib}
\DefineBibliographyStrings{ngerman}{
    andothers = {{et\,al\adddot}},             
}

\usepackage{caption}
\usepackage{float}
\usepackage{subfig}
\usepackage{xspace}
\usepackage{scrhack}

\usepackage[acronyms,toc]{glossaries} %Glossar Achtung Perl.exe muss installiert werden
\usepackage{glossaries-extra}
\usepackage{glossary-mcols}

% pdf-Optionen  -------------------------------------------------------------
\usepackage{pgfplots} % Für Kompatibilät unter verschiedenen OS
\pgfplotsset{compat=1.7}

\pdfminorversion=7 % Verwende pdf Version 1.7

\usepackage[
    bookmarks,
    bookmarksopen=true,
    bookmarksnumbered=true,
    colorlinks=true,
    %linkcolor=red, % einfache interne Verknüpfungen
    %anchorcolor=black,% Ankertext
    %citecolor=blue, % Verweise auf Literaturverzeichniseinträge im Text
    %filecolor=magenta, % Verknüpfungen, die lokale Dateien öffnen
    %menucolor=red, % Acrobat-Menüpunkte
    %urlcolor=cyan, 
    % für die Druckversion können die Farben ausgeschaltet werden:
    linkcolor=black, % einfache interne Verknüpfungen
    anchorcolor=black,% Ankertext
    citecolor=black, % Verweise auf Literaturverzeichniseinträge im Text
    filecolor=black, % Verknüpfungen, die lokale Dateien öffnen
    menucolor=black, % Acrobat-Menüpunkte
    urlcolor=black, 
    %backref,
    %pagebackref,
    plainpages=false,% zur korrekten Erstellung der Bookmarks
    pdfpagelabels,% zur korrekten Erstellung der Bookmarks
    hypertexnames=false,% zur korrekten Erstellung der Bookmarks
    linktoc=all%Sowohl Seitenzahl als auch Text als Link
]{hyperref}
\usepackage{bookmark}
\usepackage[all]{hypcap} % jump to top of figure
\usepackage{pdfpages}