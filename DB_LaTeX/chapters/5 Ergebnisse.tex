\chapter{Ergebnisse}\label{ch:Ergebnisse}

Die im Rahmen dieser Arbeit entwickelte Datenbank stellt die Basis für die weitere Arbeit an dem Verwaltungssystem für Lebensmittel. Die Tests mit dem Entwickelten Backend-Server sowie der Smartphone-App und dem \glsxtrshort{CLI}haben gezeigt, dass das Datenbankmodell auf kleiner Skala grundsätzlich für die Verwaltung der Lebensmittel von mehreren Benutzer:innen mit mehreren Geräten geeignet ist. Falls die Architektur des Backends nicht grundsätzlich verändert wird kann die Datenbank mit kleineren Modifikationen weiter verwendet werden.\\ Auch der Entwickelte Backend-Server ist grundsätzlich für die weiterentwicklung im Rahmen des weiteren Projektverlaufs geeignet. Es ist allerdings zu evaluieren, ob mit der Einführung von weiteren Funktionen die monolithische Struktur des Backends weiter bestehen soll. Es ist denkbar, dass sowohl das Benutzer:innenmanagement als auch das Gerätemanagement durch separate Services übernommen werden. Besonders für die Anbindung echter IoT-Geräte ist Möglicherweise die Verwendung einer IoT-Plattform erforderlich. In diesem Fall bietet es sich an auch die Geräteverwaltung auf dieser Plattform zu implementieren.\\ Der Entwickelte App-Prototyp kann zwar für Testzwecke weiter verwendet werden, eignet sich allerdings auf Grund der getroffenen Designentscheidungen schlecht als Ausgangspunkt für die Entwicklung der finalen App. Es wurde aufgrund der begrenzten Funktionalität des App-Prototypen auf einen sorgfältigen Entwurf der Architektur verzichtet. Wichtige Komponenten, wie eine Cache-Datenbank oder ein State-Management-System wurden nicht implementiert. Es empfiehlt sich die Entwicklung einer neuen App auf Basis eines geeigneten Templates mit einer ausgereiften Architektur.\\ Insgesamt wurden bereits wichtige Vorarbeiten für die Entwicklung des Verwaltungssystems für Lebensmittel geleistet. Auch wenn sich nicht alle Komponenten der \glsxtrshort{MVP}-Lösung für die Weiterentwicklung eignen, wurden dennoch wichtige Erkenntnisse gewonnen, die beim sorgfältigen Entwurf weiterer Produktiterationen berücksichtigt werden sollten. Besonders an Auswahl von NestJS als Backend-Framework und an der PostgreSQL-Datenbank sollte festgehalten werden.\\